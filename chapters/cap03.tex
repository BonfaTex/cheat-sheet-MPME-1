%!TEX root = ../main.tex

\vspace{-1em}

% =================================================

\section{\texorpdfstring{\color{red}Costitutive Classes}{}}

% =================================================

\textbf{Change of frame/observer.} It is just a \emph{superposed rigid body motion} $\chi^*(p,t)=r\circ \chi$, where the motion $\chi$ is
\begin{equation*}
\chi(p,t)-\chi(q,t)=F(p,t)[p-q]
\end{equation*}

and the rigid motion $r$ is
\begin{equation*}
r(x,t)-r(y,t)=Q(t)[x-y]
\end{equation*}

$\leadsto \chi^*(p,t)-\chi^*(q,t)=F^*(p,t)[p-q],\,\boxed{F^*=QF}$ \\
with $Q(t)\in\text{SO}(3)$ called \emph{frame-rotation}, and $\Omega=\dot{Q}Q^T\in\text{Skw}$ called \emph{frame-spin}. \\
In general, we can prove the followings:
\begin{equation*}
\begin{gathered}
C^*=C\ \ B^*=QBQ^T\ \ T^*=QTQ^T \\
L^*=\Omega+QLQ^T\ \ D^*=QDQ^T
\end{gathered}    \tag{$\circledast$}
\end{equation*}

\rule{0.31\textwidth}{0.2pt}
\smallskip

\textbf{Def.} $(\chi,T)$ is a \emph{mechanical process}. A \emph{costitutive class} $\Cc$ is a subset (i.e. a restriction) of the whole set of mechanical processes.

\rule{0.31\textwidth}{0.2pt}
\smallskip

\textbf{Principle of frame indifference (PFI).} "Physical laws be independent of the frame of reference" i.e. $(\chi,T)\in\Cc\ \Leftrightarrow\ (\chi^*,T^*)\in\Cc$

\rule{0.31\textwidth}{0.2pt}
\smallskip

\textbf{Material simmetry.} Let $\ti$ be a costitutive law, i.e. $T=\ti(F)$. We call
\begin{equation*}
\Gc=\left\{ H\in\text{Lin}^+\,:\,\ti(FH)=\ti(F)\ \forall\, F\in \text{Lin}^+ \right\}
\end{equation*}
the \emph{symmetry group} of the body ($\Gc$ is a group, i.e. $I\in \Gc,\ H_1,H_2,H_3\in \Gc\,\Rightarrow\,H_1H_2,H_3^{\text{-}1}\in \Gc$) \\ $\leadsto$ there are no privileged stress directions.

\smallskip 

To avoid phisically unacceptable situations (when you deform a body in a single point or you make it infinitely big) we set $\text{det}(H)=1$, i.e.
\begin{equation*}
\Gc=\left\{ H\in\text{SL}(3)\,:\,\ti(FH)=\ti(F)\ \forall\, F\in \text{Lin}^+ \right\}
\end{equation*}

We'll consider $G=\text{SL}(3)$ (fluids), $G\supseteq\text{SO}(3)$ (isotropic materials), $G\subseteq \text{SO}(3)$ (solids), $G=\text{SO}(3)$ (isotropic solids).

\smallskip 

\textbf{Thm.} Every $G$ st $\text{SO}(3)\subseteq G \subseteq \text{SL}(3)$ must either be $G=\text{SL}(3)$ or $G=\text{SO}(3)$ (there are no isotropic materials \emph{between} solids and fluids).  

\rule{0.31\textwidth}{1pt}

