%!TEX root = ../main.tex

% =================================================
% =================================================

\section{\texorpdfstring{\color{red}Linear Elasticity}{}}

% =================================================
% =================================================

Here the focus is on the displacement $\underline{u}=\underline{x}-\underline{X}$. Since $\underline{x}(\underline{X},t)=\underline{X}+\underline{u}(\underline{X},t)$, the deformation gradient is $\boxed{F=I+\nabla_{\!\!_X} \underline{u}}$.  

\rule{0.31\textwidth}{0.2pt}
\smallskip

We consider $\underline{u}$ and $\nabla \underline{u}$ as small quantities, i.e. $\underline{u}=o(\Ec)=\nabla\underline{u}$, and we neglet terms of $o(\Ec^2)$.

\smallskip

In this framework, $C$ (also $B$) becomes
\vspace{-0.2em}
\begin{equation*}
\big(I+(\nabla \underline{u})^T\big) \big(I+\nabla \underline{u}\big)=I+2E+o(\Ec^2)
\end{equation*}
with $E:=\text{sym}\,\nabla\underline{u}$ \emph{infinitesimal strain tensor}.

\rule{0.31\textwidth}{0.2pt}
\smallskip

The elastic law $T=\hat{T}(F)$ becomes
\begin{equation*}
\hat{T}(I+\nabla\underline{u})=\underbracket[0.5pt]{\hat{T}(I)}_{:=T_0}+\underbracket[0.5pt]{D_F\hat{T}(I)}_{:=\CC}\big[\nabla\underline{u}\big]+o(\nabla\underline{u}) 
\end{equation*}
that is the \emph{infinitesimal elasticity law}.

\smallskip

However, we assume $T_0=0$ (i.e. no pre-stress), thus we have
\begin{equation*}
\boxed{T=\CC\big[\nabla\underline{u}\big]}
\end{equation*}

the \emph{linear elasticity law}.

\rule{0.31\textwidth}{0.2pt}
\smallskip

The \emph{linear elastic tensor} $\CC$ is
\begin{align*}
\CC:=D_F\hat{T}(I):\text{Lin}&\to\text{Sym} \\
\nabla\underline{u}&\mapsto \CC\big[\nabla\underline{u}\big]=T
\end{align*}

Since $T$ is symmetric i.e. $T_{ij}=T_{ji}$ then
\begin{equation*}
C_{ijhk}=\left.\frac{\partial T_{ij}}{\partial F_{hk}}\right|_{F_{hk}=I_{hk}} =C_{jihk}
\end{equation*}

Moreover, from (PFI) and $T_0=0$ one can deduce $\CC\big[\Omega\big]=0$ for every $\Omega\in\text{Skw}$, implying that
\begin{gather*}
\CC\big[\nabla\underline{u}\big]=\CC\big[\text{sym}\,\nabla\underline{u}\big]+\cancel{\CC\big[\text{skw}\,\nabla\underline{u}\big]}=\CC[E] \\
\leadsto\ \boxed{
\begin{aligned}
\CC:\text{Sym}&\to\text{Sym} \\
E&\mapsto T
\end{aligned}
\qquad C_{jihk}=C_{ijhk}=C_{ijkh}
}
\end{gather*}

\rule{0.31\textwidth}{0.2pt}
\smallskip

$T$ and $S$ are equivalent (negligible difference).

\rule{0.31\textwidth}{0.2pt}

% =================================================

\subsection{\texorpdfstring{\color{red}Isotropic Linear Elasticity}{}}

% =================================================

\textbf{Costitutive law and simmetry group.} \\
$\qquad T=\hat{T}(F)=\CC[E]\qquad G=\text{SO}(3)$

\rule{0.31\textwidth}{0.2pt}
\smallskip

\textbf{PFI.} $\boxed{Q\CC[E]Q^T=\CC[QEQ^T]\quad \forall\,Q\in\text{SO}(3)}$

\rule{0.31\textwidth}{0.2pt}
\smallskip

\textbf{Thm (representation).} 
\begin{equation*}
\boxed{\CC[E]=\lambda(\text{tr}\,E)I+2\mu E}
\end{equation*}
where $\lambda,\,\mu$ are the Lamé constants.

\rule{0.31\textwidth}{0.2pt}
\smallskip

\textbf{Balance equations.} From $\rho_*\ddot{\underline{x}}=\text{Div}\,S+\underline{b}_*$ we obtain the Navier equation
\begin{equation*}
\boxed{\rho_*\ddot{\underline{u}}=(\lambda+\mu)\nabla\big(\text{Div}\,\underline{u}\big)+\mu\,\Delta \underline{u}+\underline{b}_*}
\end{equation*}

\rule{0.31\textwidth}{1pt}









