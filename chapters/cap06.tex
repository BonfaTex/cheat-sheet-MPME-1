%!TEX root = ../main.tex

\vspace{-1em}

% =================================================
% =================================================

\section{\texorpdfstring{\color{red}Hyperelasticity}{}}

% =================================================
% =================================================

\textbf{Def.} An elastic body is hyperelastic if there exists an \emph{elastic energy density per unit volume}
\begin{equation*}
\omega:\text{Lin}^+\to\RR\,, \qquad F\mapsto \omega(F)  
\end{equation*}
such that $\boxed{S=\nicefrac{\partial\omega}{\partial F}}$

\smallskip

$\leadsto\ S\cdot \dot{F}=\frac{\de\ \omega\big(F(\underline{x},t)\big)}{\de t}$, so $\Pi^{\,\text{int}}=-\frac{\de }{\de t}\int_{_{\pt_*}}\!\!\omega\ \de V_{_X}$

\rule{0.31\textwidth}{0.2pt}
\smallskip

\textbf{Thm (characterization of hyperelastic body).} $\exists\,\omega$ iff $ \oint_{t_0}^{t_1} S\cdot\dot{F}\dt=0$
for any mechanical process; this implies $\int S\cdot\dot{F}\dt$ depends only on the extremes.

\rule{0.31\textwidth}{0.2pt}
\smallskip

\textbf{Potential elastic energy.} $\Uc(\pt_*)=\int_{_{\pt_*}}\!\!\omega\ \de V_{_X}$

\smallskip

By $(\propto\!m)$, we know $\Pi^{\,\text{int}}=-\dot{\Uc}$, thus the kinetic energy thm becomes $\dot{T_k}+\dot{\Uc}=\Pi^{\,\text{ext}}$. Moreover, if external forces are conservative, i.e. $=\frac{\de \Vc}{\de t}$, then the thm simply is $\frac{\de}{\de t}\ (T_k+\Uc+\Vc)=0$.

\rule{0.31\textwidth}{0.2pt}
\smallskip 

One can rewrite using $\sigma(F)$ \emph{energy density per unit mass} st $\omega(F)=\rho_*\sigma(F)$, for istance
\vspace{-0.5em}
\begin{equation*}
\Uc=\int_{_{\pt_*}}\!\!\!\!\omega\ \de V_{_X}=\int_{_{\pt_*}}\!\!\!\!\rho_*\sigma\ \de V_{_X}=\int_{_{\pt_*}}\!\!\!\!\rho\,\sigma\ \de V_x
\end{equation*}
\vspace{-0.5em}
\rule{0.31\textwidth}{0.2pt}

\newcolumn

\textbf{PFI.} $\omega(F)=\omega(QF)$ $\forall Q\in\text{SO}(3)$

\smallskip

\textbf{Thm.} (PFI) $\Longleftrightarrow$ $T$ symmetric.

\rule{0.31\textwidth}{0.2pt}
\smallskip

Thanks to polar decomposition thm
\vspace{-1em}

\begin{equation*}
\omega(F)=\omega(QF)=\omega(QRU)\overset{Q=R^T}{=}\omega(U)
\end{equation*}
and $C=F^TF=U^2$ $\leadsto$ $\boxed{\omega=\hat{\omega}(C)}$

\smallskip

The advantage is that in this form the (PFI) is automatically satisfied, and in addiction
\begin{equation*}
\boxed{S=2F\,\textstyle\frac{\partial\,\hat{\omega}}{\partial C}}
\end{equation*}

and $T$ follows from $T=J^{\text{-}1}SF^T$.

\smallskip

With isotropic incompressible hyperel. materials
\begin{align*}
S&=-pF^{\,\text{-}T}+\textstyle\frac{\partial\,\omega}{\partial F} \\  
T&=-p\,I+\textstyle\frac{\partial\,\omega}{\partial F}\,F^T
\end{align*}


\rule{0.31\textwidth}{1pt}














